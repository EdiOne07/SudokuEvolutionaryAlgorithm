\section{Introduction}
\label{sec:intro}

Traditionally, a Sudoku is a logic based number placement puzzle. The objective of the puzzle is to fill a most commonly 9x9 sized grid with digits so that each column, row and 3x3 subgrid contain all of the digits from 1 to 9.\cite{}

Solving such a puzzle programmatically falls into the category of search and optimization problems. These types of problems can be approached in different ways. One possible approach is depicted by evolutionary algorithms or in this case more specifically, genetic algorithms (GA)\cite{WikiSudoku}.

Inspired by the process of natural selection, GAs use biologically inspired operators such as selection, crossover and mutation to generate solutions to optimization and search problems\cite{WikiGA}.

This paper studies the efficiency of solving Sudoku puzzles with GA approaches. The objective is to compare the performance of different implementations of GAs. These approaches will additionally be compared to the naive search algorithm depth-first search (DFS). The puzzles explored will be of different difficulties for regular 9x9 grids and bigger 16x16 and 25x25 grids. This way the paper analyzes whether GAs are a viable option for efficiently solving Sudoku boards compared to naive search algorithms as the complexity of the puzzle increases.

This document is organized in the following way. Chapter one describes the objective of this study. Chapter two describes the problem of solving Sudoku puzzles and the motivation of using evolutionary algorithm approaches to solve this type of problem. Chapter three details the algorithm used in this study. The following chapter lays out the setup of the experiments run in this study. Lastly, chapter five presents and analyses the results of this paper.