\section{Introduction}
\label{sec:intro}

\textcolor{red}{The recommended software for typesetting assignment reports is \LaTeX. It will allow you to prepare high-quality documents, especially in the area of Computer Science. This document can serve as a template for reports. Each section begins with brief instructions in red text. All the instructions in red, as well as the dummy text, should be removed in the final version to submit. The \LaTeX\ source of this file includes examples of using the most needed commands and environments. You can find plenty of other examples with explanations in many web forums and discussion groups on the Internet. The easiest way to edit your report is to use \url{https://www.overleaf.com/}. Overleaf does not require any setup on your computer, and it is free to create an account.}

\textcolor{red}{The book \textit{Writing for Computer Science} \cite{zobel2014writing} is a useful assistance on how to write properly and present your work when it comes to Computer Science topics. It is a strong recommendation to follow its guidelines and limit the usage of AI tools to generate text. Keep in mind that the examiner is an expert in Evolutionary Computation and therefore, any false information generated by an AI tool is easily notable. Such case may lead to failing the assignment.}

\textcolor{red}{The introduction should briefly introduce the assignment and its purpose.}
 
Traditionally, a Sudoku is a logic based number placement puzzle. The objective of the puzzle is to fill a most commonly 9x9 sized grid with digits so that each column, row and 3x3 subgrid contain all of the digits from 1 to 9.\cite{}

Solving such a puzzle programmatically falls into the category of search and optimization problems. These types of problems can be approached in different ways. One possible approach is depicted by evolutionary algorithms or in this case more specifically, genetic algorithms (GA)\cite{WikiSudoku}.

Inspired by the process of natural selection, GAs use biologically inspired operators such as selection, crossover and mutation to generate solutions to optimization and search problems\cite{WikiGA}.

This paper studies the efficiency of solving Sudoku puzzles with GA approaches. The objective is to compare the performance of different implementations of GAs. These approaches will additionally be compared to the naive search algorithm depth-first search (DFS). The puzzles explored will be of different difficulties for regular 9x9 grids and bigger 16x16 and 25x25 grids. This way the paper analyzes whether GAs are a viable option for efficiently solving Sudoku boards compared to naive search algorithms as the complexity of the puzzle increases.

This document is organized in the following way. Chapter one describes the objective of this study. Chapter two describes the problem of solving Sudoku puzzles and the motivation of using evolutionary algorithm approaches to solve this type of problem. Chapter three details the algorithm used in this study. The following chapter lays out the setup of the experiments run in this study. Lastly, chapter five presents and analyses the results of this paper.