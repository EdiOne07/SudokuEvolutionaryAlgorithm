\section{Introduction}
\label{sec:intro}

\textcolor{red}{The recommended software for typesetting assignment reports is \LaTeX. It will allow you to prepare high-quality documents, especially in the area of Computer Science. This document can serve as a template for reports. Each section begins with brief instructions in red text. All the instructions in red, as well as the dummy text, should be removed in the final version to submit. The \LaTeX\ source of this file includes examples of using the most needed commands and environments. You can find plenty of other examples with explanations in many web forums and discussion groups on the Internet. The easiest way to edit your report is to use \url{https://www.overleaf.com/}. Overleaf does not require any setup on your computer, and it is free to create an account.}

\textcolor{red}{The book \textit{Writing for Computer Science} \cite{zobel2014writing} is a useful assistance on how to write properly and present your work when it comes to Computer Science topics. It is a strong recommendation to follow its guidelines and limit the usage of AI tools to generate text. Keep in mind that the examiner is an expert in Evolutionary Computation and therefore, any false information generated by an AI tool is easily notable. Such case may lead to failing the assignment.}

\textcolor{red}{The introduction should briefly introduce the assignment and its purpose.}

---

- This paper tries to find an Evolutionary Algorithm approach to solve scalable (bigger than standard 9x9) Sudokus. 
(is this just a Genetic Algorithm?)

- It discusses why naive methods fail (hopefully) as the size of the puzzle increases and why an evolutionary algorithm approach is more scalable in the context of solving Sudokus

- The objective is to minimize the time the evolutionary algorithm takes to solve the Sudokus of increasing sizes.

- The baseline to compare the performance of the evolutionary algorithm to will be Depth-First Search(DFS). (Maybe add more naive methods or the very basic evolutionary algorithm without any modifications)

- The document is structured as follows:
Section 1: Introduction and goal of the paper.
Section 2: Describes the problem the paper solves in detail + motivation of the evolutionary approach used.
Section 3: Describes the algorithm in detail.
Section 4: Describes the experiments run with the algorithm.
Section 5: Presents and analyses the results.
Section 6: Conclusion.

