\section{Sudoku Puzzle Problem}
\label{sec:problem_description}
\subsection{Problem Description}
{Sudoku is a Japanese logical game that is played on a $9 \times 9$ grid, that are further divided into $3 \times 3$ subgrids. The objective of the game is to fill the grid with digits from 1 to 9, ensuring that each row, column, and subgrid contains each digit exactly once\cite{Mantere2007}. The puzzle starts with some cells already filled in, and the player must use logic and respect the base rule of the game in order to finish the puzzle.}
{\newline}
{\newline Sudoku puzzles can vary in difficulty based on how many numbers they start with, the arrangement of these numbers and even the varying size of the sudoku, since they can go beyond the standard $9 \times 9$ grid, similar to the $25 \times 25$ grid we used to test our algorithm with. Thus we can think of the sudoku problem as a graph coloring problem if it were to be expressed in a mathematical context. The $9 \times 9$ grid can be seen as graph that has 81 vertices, namely one vertex for each cell. Each vertex can be labeled with an ordered pair (x,y), where x and y are integers between 1 and 9 \cite{WikiMathematics}. Two distinct vertices (x1,y1) and (x2,y2) are connected by an edge if and only if they are in the same row, column or subgrid:}
\begin {itemize}
    \item $x1 = x2$, which translates as same row
    \item $y1 = y2$, which translates as same column
    \item $\lfloor (x1-1)/3 \rfloor = \lfloor (x2-1)/3 \rfloor$ and $\lfloor (y1-1)/3 \rfloor = \lfloor (y2-1)/3 \rfloor$ which translates as same subgrid
\end {itemize}
{As previously mentioned, the puzzle is completed when all the vertices have an integer between 1 and 9 assigned to them, in such a way that vertices that are joined by an edge don't have the same number assigned to them. }
\subsection{Evolutionary Approach}
{While Sudoku has a deterministic solution, evolutionary algorithms can be used in order to compare their performance with more traditional approaches such as the depth first search algorithm. As it is mentioned in\cite{Moraglio}, they managed to implement an evolutionary algorith that could solve very efficently easy Sudoku puzzles. The downside of their algorithm, was that it wasn't as efficient when it came down to solving medium or hard puzzles.   }
{\color{red}
\begin{itemize}
    \item The mathematical formulation considered in your study. Some problems have a clear mathematical model (e.g., Travelling Salesman Problem), while others do not (e.g., $n$-Queens). Based on the problem you chose, search the literature and find a proper way to present the problem.
    \item One paragraph that briefly presents at least 3 published academic works where any evolutionary approach is used to solve the problem. It would be wise to cite here works that influenced your algorithm. This practice saves you time from looking for additional academic resources. You can find more information about reading and searching in the literature in \cite{zobel2014reading}.
    \item The motivation behind the evolutionary approach you decided to develop. A good practice would be to align the motivation with some literature gap found in the academic works you presented above. However, this is not mandatory. You can motivate your selection on the characteristics of the algorithm making it proper for the problem.
\end{itemize}
}

\textcolor{red}{\textbf{Note:} Change the section's title to match the name of the problem you chose for your assignment.}

---

- This section presents the problem this paper solves with the use of evolutionary algorithm approach. 

- Math in solving Sudokus? Maybe rather introducing quickly a sudoku, how to solve it and describing the steep increase of difficulty with increasing sizes of sudoku puzzles

- Cite 3 published papers. Cite where our idea is from\cite{Mantere2007}. Another Paper\cite{Amil2019}.

- Motivation behind evolutionary approach.


