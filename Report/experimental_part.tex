\section{Experimental part}
\label{sec:experimentation}

{\color{red}
This section describes the setup of experiments \cite{zobel2014experimentation}:

\begin{itemize}
    \item Provide the details of the hardware and software that you used.
    \item Describe the steps you carried out during your experiments.
    \item Detail the data you used for the evaluation of your algorithm.
\end{itemize}
}

Hardware: Macbook pro M1 max
Software: Python (VS Code)
----
we have algorithm
now we want hyperparamter -> bayesian optimization (paper)
give us best population and mutation rate
with these hp we run the algrothm easy, medium, hard and 16x16 and 25x25
we track execution time, number of generations to solution (or max of 10.000 gens with best fitness, no gen without improvement)
Run default DFS and track time and solution 

----
random boards for data
using DFS we generate a solved board for the desired size
randomly remove number and try to solve again. repeat, if we cannot solve it we restart
9x9 remove until difficulty is met (easy: range of 40.50, med: 30-39, hard 25-29) (cite something?)
16x16: from the internet?
25x25: from internet? ask Eddie (where from)
big ones taken because generating them takes a lot of time, only test with one board because of complexity

----------------------

This section describes the setup of the experiments run to study the efficiency of solving Sudoku puzzles with GA approaches.

The test will be run on a Macbook pro M1 max.

\subsection{Choosing hyperparameters}
Chapter 3 discusses the algorithm used for the experiments in detail. However, before the algorithm can be run, the hyperparameters for \textit{\textbf{population\_size}}, \textit{\textbf{mutation\_rate}} and \textit{\textbf{elite\_size}} need to be chosen. The solution quality of a stochastic algorithm strongly depends on choosing the correct hyperparameters. Previous studies show that configuring hyperparameters of GAs using Bayesian Optimization leads to significantly better results than choosing them at random while keeping the computational time low\cite{Ruether}.

The Bayesian Optimization evaluates the optimal hyperparamters of the GA as follows...

\subsection{Running the experiment}
Next the GA is run 1000 times for a maximum of 10.000 generations for Sudoku puzzles of easy, medium and hard difficulty of 9x9 boards. Because of the drastic jump in complexity with 16x16 and 25x25 boards, the GA is only run ? times for these puzzles. For the evaluation of the performance the following properties of the experiment are tracked:
\begin{enumerate}
	\item Number of successfully solved boards
	\item Execution time
	\item Number of average generations
	\item Average execution time per Sudoku puzzle
	\item Average generations per second
\end{enumerate}

For failed runs the following properties are tracked:
\begin{enumerate}
	\item Number of generations stuck at local minimum
	\item Best fitness achieved
	\item Average violations
	\item Generations without improvement
	\item Average generations without improvement
\end{enumerate}

To compare the performance of the GA, a DFS algorithm is run on Sudoku boards of the same complexity. For the evaluation of the performance the following properties of the experiment are tracked:
\begin{enumerate}
	\item Number of successfully solved boards
	\item Execution time
	\item Average execution time per Sudoku puzzle
\end{enumerate}

\subsection{Sudoku boards used in the experiment}
Because of the high number of Sudoku boards needed to accurately evaluate the performance of the stochastic algorithm, the Sudoku grids are generated randomly as part of the experiment. The boards are generated in the following way. 
First, a solved board is generated using the DFS algorithm. Next, a random number is removed from the grid. The DFS algorithm tries to solve the puzzle again. If it is still solvable, repeat the previous steps until the requirement of given numbers is met for the different complexities of the Sudoku boards. For 9x9 grids, the easy difficulty gives between 40-50 numbers. Medium and hard puzzles have 30-39 and 25-29 givens respectively. (source required?)

Because of the increased complexity, run time and reduced success rate of solving 16x16 and 25x25 puzzles, only ? boards were taken from...